\documentclass[13pt]{article}
\usepackage[fontsize=10pt]{fontsize}

\usepackage[margin=0.5in]{geometry} 
\usepackage{amsmath,amsthm,amssymb, graphicx, multicol, array, txfonts}
\usepackage{bbm}
\usepackage{hyperref}
\hypersetup{
    colorlinks=true,
    linkcolor=blue,
    filecolor=magenta,      
    urlcolor=cyan,
    pdftitle={Overleaf Example},
    pdfpagemode=FullScreen,
    }

\urlstyle{same}


\newcommand{\N}{\mathbb{N}}
\newcommand{\Z}{\mathbb{Z}}
\setcounter{secnumdepth}{0}
\setlength\parindent{0pt}

 
\newenvironment{problem}[2][Problem]{\begin{trivlist}
\item[\hskip \labelsep {\bfseries #1}\hskip \labelsep {\bfseries #2.}]}{\end{trivlist}}

\newenvironment{prelim}[2][Preliminaries]{\begin{trivlist}
\item[\hskip \labelsep {\bfseries #1}\hskip \labelsep {\bfseries #2}]}{\end{trivlist}}
    
\begin{document}
 
\title{6.S088 Problem Set 2}
\author{Suyeol Yun\\
syyun@mit.edu}
\maketitle
 
\section{Problem 3}
\begin{align*}
    K\left(\mathbf{x}, \mathbf{x}^{\prime}\right)=e^{-L\left\|\mathbf{x} - \mathbf{x}^{\prime}\right\|^2}=e^{-L\|\mathbf{x}\|^2} e^{-L\left\|\mathbf{x}^{\prime}\right\|^2} e^{2L\left\langle\mathbf{x}, \mathbf{x}^{\prime}\right\rangle} && (1)
\end{align*}

For the term $e^{2L\left\langle\mathbf{x}, \mathbf{x}^{\prime}\right\rangle}$, we use Taylor expansion of $e^z$ around $z=0$ and geometry

\begin{align*}
    e^{2L\left\langle\mathbf{x}, \mathbf{x}^{\prime}\right\rangle}=\sum_{k=0}^{\infty} \frac{1}{k !}\left(2L\left\langle\mathbf{x}, \mathbf{x}^{\prime}\right\rangle\right)^k && (2)
\end{align*}

Then we have 

\begin{align*}
    \left\langle\mathbf{x}, \mathbf{x}^{\prime}\right\rangle^k=\left(\sum_{i=1}^d \mathbf{x}_i \mathbf{x}_i^{\prime}\right)^k=\sum_{j \in[d]^k}\left(\prod_{i=1}^k \mathbf{x}_{j_i}\right)\left(\prod_{i=1}^k \mathbf{x}_{j_i}^{\prime}\right) && (3)
\end{align*}

where $j$ enumerates over all selections of $k$ coordinates of $\mathbf{x}$. Then by plugging in $(2), (3)$ to $(1)$ results in

\begin{align*}
    K\left(\mathbf{x}, \mathbf{x}^{\prime}\right)&=\left\langle\phi(\mathbf{x}), \phi\left(\mathbf{x}^{\prime}\right)\right\rangle=\prod_{k=0}^{\infty} \prod_{j \in[d]^k} \phi_{k, j}(\mathbf{x}) \phi_{k, j}\left(\mathbf{x}^{\prime}\right)\\
    &\text{ where }     \phi_{k, j}(\mathbf{x})=e^{-L\|\mathbf{x}\|^2} \frac{2L^{\frac{K}{2}}}{\sqrt{k !}} \prod_{i=0}^k \mathbf{x}_{j_i}
\end{align*}

\section{Problem 4}

\begin{align*}
\sum_{i=1}^n \sum_{j=1}^n c_i c_j K\left(x^{(i)}, x^{(j)}\right) &= \sum_{i=1}^n \sum_{j=1}^n c_i c_j \langle\psi(x^{(i))}, \psi(x^{(j)})\rangle_{\mathcal{H}}\\
&= \left(\sum_{i=1}^n c_i \psi(x^{(i)}) \right)^T\left(\sum_{i=1}^n c_i \psi(x^{(i)}) \right) \ge 0\\
\end{align*}

$\sum_{i=1}^n \sum_{j=1}^n c_i c_j K\left(x^{(i)}, x^{(j)}\right) \ge 0$ implies $K$ is \textit{positive semi-definite}.

\end{document}